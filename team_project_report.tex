\documentclass{article}
% \renewcommand{\familydefault}{\sfdefault}
% \usepackage{fontspec}
% \defaultfontfeatures{Mapping=tex-text,Scale=MatchLowercase}
% \setmainfont{Arial}
% \setlength{\headheight}{12.5pt}
% \usepackage[utf8]{inputenc}
\usepackage{cmbright}
\usepackage{geometry}
\usepackage{amsmath}
\usepackage{float}
\usepackage{diagbox}
 \geometry{
 a4paper,
 total={170mm,270mm},
 left=20mm,
 top=10mm,
 }
 \usepackage{graphicx}
 \usepackage{titling}

 \title{MF731 Team Project Report
}
\author{Che Wang, Liao Huaze, Liu Yuqing, Xizhi Wang}
\date{November 2022}
 
%  \usepackage{fancyhdr}
% \fancypagestyle{plain}{%  the preset of fancyhdr 
%     \fancyhf{} % clear all header and footer fields
%     \fancyhead[L]{Description of Assignment}
% }
\makeatletter
\def\@maketitle{%
  \newpage
  \null
  \vskip 1em%
  \begin{center}%
  \let \footnote \thanks
    {\LARGE \@title \par}%
    \vskip 1em%
    % {\large \@date}%
  \end{center}%
  \par
  \vskip 1em}
\makeatother
% -------------------------------------------------------
% -------------------------------------------------------
% -------------------------------------------------------
\begin{document}

\maketitle

\begin{center}
\noindent\begin{tabular}{@{}ll}
        Team Members: & \theauthor\\
\end{tabular}
\end{center}
\section*{Executive Summary}
We started our project research with a meeting discussing the targets of our research, problems that we would face, as well as the steps that we need to take. 

After figuring out all the questions that we had and having a general idea about how to carry out our research, We divided the research into 4 parts. Che Wang is responsible for question 4-5, 14-15 of the research. Huaze Liao is responsible for question 6-7, 12-13. Yuqing Liu is responsible for question 8-11. Xizhi Wang is responsible for 16-17, and he also contributed a lot to helping our members. 

After finishing writing our python code, we had another meeting to review each other's code and put them together to form our final solution. 
\section*{Methodology}
\subsection*{Point 4-5}
For point 4, we wrote a function to simulate $\tilde {X_1}(i,j)$ and $\tilde {X_2}(i,j)$ and loop this function 100,000 times to generate 100,000 simulations. 
Within this function, we first generate a $Z$ matrix, the shape of which is 5 times 5, and each element of the matrix is the same number sampled from the standard normal distribution. Then we generate a $Z(i,j)$ matrix, which also has the shape of 5 times 5, and each element of the matrix is a different sample from the standard normal distribution. After that we use the following formula to simulate $\tilde {X_1}(i,j)$, $\tilde {X_2}(i,j)$ and $\tilde X$.
\begin{equation}
  \begin{aligned}
\tilde {X_1}(i,j) = 0.3Z + \sqrt{1-0.3^2} Z(i,j), \quad\tilde {X_1}(i,j) \sim N (0.004,0.04)\\
\tilde {X_2}(i,j) = 0.1Z + \sqrt{1-0.1^2} Z(i,j), \quad\tilde {X_2}(i,j) \sim N (0.006,0.05)\\
\notag
  \end{aligned}
\end{equation}
Then we wrote two functions to calculate the VaR and ES.
VaR, ES is calculated with the formulas:
$$
VaR_\alpha = l_{(\lceil n\alpha \rceil)}, \quad
ES_{\alpha}(F_n^e) = \frac {1} {n(1-\alpha)} \sum ^n_{k=\lceil n\alpha \rceil + 1} l_{(k)} + \frac {\lceil n\alpha \rceil - n\alpha} {n(1-\alpha)} l_{(\lceil n\alpha \rceil)}
\notag
$$

\noindent For point 5, we follow the instructions and set $\epsilon = 0.005$, with $\tilde {X_n} \in (VaR_{0.99} - \epsilon, VaR_{0.99} + \epsilon)$, take average of $\tilde {X_1}$ and $\tilde {X_2}$ to find the Euler allocations of
VaR0.99 for the risk position 1 and 2.
With $\tilde {X} \leq VaR_{0.975}$,  take average of $\tilde {X_1}$ and $\tilde {X_2}$ for these simulations to find the Euler allocations of $ES_{0.975}$ for the risk position 1 and 2.
\subsection*{Point 6-7}
\subsection*{Point 8-11}
\subsection*{Point 12-13}
\subsection*{Point 14-15}
For point 14, we calculated the $ES^{F,C}(X_n(6,j)|X(6))$ using the same method as point 13.
\\For point 15, with the assumption in item 8 that $\frac {ES^{R,S}(X(i))} {ES^{R,C}(X(i))} = 2$, we have $IMCC(X_n(i,j)|X(i))=ES^{F,C}(X_n(i,j)|X(i))$.
\subsection*{Point 16-17}
\section*{Result}

\subsection*{Point 4-5}
For point 4, we found that the $VaR_{0.99}$ for the 10 days loss of the portfolio is around: 5.26 and the $ES_{0.975}$ for the 10 days loss of the portfolio is around: 5.27.
\\For point 5, the Euler allocation of $VaR_{0.99}$ for the risk position 1 is around: 3.174, and the Euler allocation of $VaR_{0.99}$ for the risk position 2 is around: 2.088. The Euler allocation of $ES_{0.975}$ for the risk position 1 is around: 3.435, and the Euler allocation of $ES_{0.975}$ for the risk position 2 is around: 1.836.
\subsection*{Point 6-7}
\subsection*{Point 8-11}
\subsection*{Point 12-13}
\subsection*{Point 14-15}
For point 14:\\ $ES^{F,C}(X_1(6,j)|X(6))$ is:
\begin{table}[H]
  \label{tab:my-table}
  \resizebox{\columnwidth}{!}{%
  \begin{tabular}{llllll}
  \hline
  j                         & 1          & 2          & 3         & 4          & 5          \\ \hline
  $ES^{F,C}(X_1(6,j)|X(6))$ & 1.75743797 & 1.20043151 & 1.4947551 & 0.79798473 & 0.89325796 \\ \hline
  \end{tabular}%
  }
  \end{table}
\noindent$ES^{F,C}(X_2(6,j)|X(6))$ is:
\begin{table}[H]
  \label{tab:my-table2}
  \resizebox{\columnwidth}{!}{%
  \begin{tabular}{llllll}
  \hline
  j                         & 1          & 2          & 3         & 4          & 5          \\ \hline
  $ES^{F,C}(X_1(6,j)|X(6))$ & 0.93965622 & 0.70802632 & 1.00024164& 0.6183377  & 0.89227294 \\ \hline
  \end{tabular}%
  }
  \end{table}
\noindent For point 15:\\
\noindent $IMCC(X_1(i,j)|X(i))$ is:
\begin{table}[H]
  \label{tab:my-table3}
  \resizebox{\columnwidth}{!}{%
  \begin{tabular}{llllll}
  \hline
  \diagbox{i}{j}     & 1          & 2          & 3         & 4          & 5          \\ \hline
  1&0.41115926&0.31388585&0.43034273&0.26614056&0.35988107 \\ 
  2&0.41681235&0.30762569&0.42334948&0.2582787 &0.35317547 \\ 
  3&0.41331437&0.30808245&0.4282395 &0.2622814 &0.3617308  \\ 
  4&0.41561675&0.30987268&0.43425611&0.26626251&0.36629749 \\ 
  5&0.43032133&0.31602653&0.43956077&0.26155237&0.35826667 \\ 
  6&1.75743797&1.20043151&1.4947551 &0.79798473&0.89325796 \\ \hline
  \end{tabular}%
  }
  \end{table}

\noindent $IMCC(X_2(i,j)|X(i))$ is:
\begin{table}[H]
  \label{tab:my-table3}
  \resizebox{\columnwidth}{!}{%
  \begin{tabular}{llllll}
  \hline
  \diagbox{i}{j}     & 1          & 2          & 3         & 4          & 5          \\ \hline
  1&0.40297301&0.31898195&0.46850977&0.30554344&0.45115918 \\
  2&0.40679785&0.32162876&0.47768414&0.30900191&0.45527033 \\
  3&0.40467146&0.31101453&0.46934114&0.30973559&0.44907883 \\
  4&0.40330508&0.32178986&0.46947583&0.30531772&0.44742337 \\
  5&0.40387888&0.31597148&0.45994035&0.30576853&0.44302283 \\
  6&0.93965622&0.70802632&1.00024164&0.6183377 &0.89227294 \\ \hline
  \end{tabular}%
  }
  \end{table}
\subsection*{Point 16-17}

\end{document}
