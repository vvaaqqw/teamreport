\documentclass{article}
% \renewcommand{\familydefault}{\sfdefault}
% \usepackage{fontspec}
% \defaultfontfeatures{Mapping=tex-text,Scale=MatchLowercase}
% \setmainfont{Arial}
% \setlength{\headheight}{12.5pt}
% \usepackage[utf8]{inputenc}
\usepackage{cmbright}
\usepackage{geometry}
\usepackage{amsmath}
\usepackage{float}
\usepackage{diagbox}
 \geometry{
 a4paper,
 total={170mm,270mm},
 left=20mm,
 top=10mm,
 }
 \usepackage{graphicx}
 \usepackage{titling}

 \title{MF731 Team Project Report
}
\author{Che Wang, Liao Huaze, Liu Yuqing, Xizhi Wang}
\date{November 2022}
 
%  \usepackage{fancyhdr}
% \fancypagestyle{plain}{%  the preset of fancyhdr 
%     \fancyhf{} % clear all header and footer fields
%     \fancyhead[L]{Description of Assignment}
% }
\makeatletter
\def\@maketitle{%
  \newpage
  \null
  \vskip 1em%
  \begin{center}%
  \let \footnote \thanks
    {\LARGE \@title \par}%
    \vskip 1em%
    % {\large \@date}%
  \end{center}%
  \par
  \vskip 1em}
\makeatother
% -------------------------------------------------------
% -------------------------------------------------------
% -------------------------------------------------------
\begin{document}

\maketitle

\begin{center}
\noindent\begin{tabular}{@{}ll}
        Team Members: & \theauthor\\
\end{tabular}
\end{center}
\section*{\underline{Executive Summary}}
We started our project research with a meeting discussing the targets of our research, problems that we would face, as well as the steps that we need to take. 

After figuring out all the questions that we had and having a general idea about how to carry out our research, We divided the research into 4 parts. Che Wang is responsible for question 4-5, 14-15 of the research. Huaze Liao is responsible for question 6-7, 12-13. Yuqing Liu is responsible for question 8-11. Xizhi Wang is responsible for 16-17, and he also contributed a lot to helping other members. 

After finishing writing our python code, we had another meeting to review each other's code and put them together to form our final solution. 
\section*{\underline{Methodology}}
\subsection*{Step 4-5}
For step 4, we wrote a function to simulate $\tilde {X_1}(i,j)$ and $\tilde {X_2}(i,j)$ and loop this function 1,000,000 times to generate 100,000 simulations. 
Within this function, we first generate a $Z$ matrix, the shape of which is 5 times 5, and each element of the matrix is the same number sampled from the standard normal distribution. Then we generate a $Z(i,j)$ matrix, which also has the shape of 5 times 5, and each element of the matrix is a different sample from the standard normal distribution. After that we use the following formula to simulate $\tilde {X_1}(i,j)$. Then use the same method to simulate $\tilde {X_2}(i,j)$.
\begin{equation}
  \begin{aligned}
\frac {\tilde {X_1}(i,j)-0.004} {0.04} = \sqrt{0.3}Z+\sqrt{0.7}Z(i,j) , \quad\tilde {X_1}(i,j) \sim N (0.004,0.04)\\
\frac {\tilde {X_2}(i,j)-0.006} {0.05} = \sqrt{0.1}Z+\sqrt{0.9}Z(i,j) , \quad\tilde {X_2}(i,j) \sim N (0.006,0.05)\\
\notag
  \end{aligned}
\end{equation}
We wrote two functions to calculate the VaR and ES.
VaR, ES is calculated with the formulas:
$$
VaR_\alpha = l_{(\lceil n\alpha \rceil)}, \quad
ES_{\alpha}(F_n^e) = \frac {1} {n(1-\alpha)} \sum ^n_{k=\lceil n\alpha \rceil + 1} l_{(k)} + \frac {\lceil n\alpha \rceil - n\alpha} {n(1-\alpha)} l_{(\lceil n\alpha \rceil)}
\notag
$$

\noindent For step 5, we follow the instructions and set $\epsilon = 0.005$, with $\tilde {X_n} \in (VaR_{0.99} - \epsilon, VaR_{0.99} + \epsilon)$, take average of $\tilde {X_1}$ and $\tilde {X_2}$ to find the Euler allocations of
VaR0.99 for the risk position 1 and 2.
With $\tilde {X} \leq VaR_{0.975}$,  take average of $\tilde {X_1}$ and $\tilde {X_2}$ for these simulations to find the Euler allocations of $ES_{0.975}$ for the risk position 1 and 2.
\subsection*{Step 6-7}
Step 6:\\
From step 4 we get the simulated loss for risk position 1, 2. Then we use the function “calc\_LH\_adjed\_loss” to transform these simulated loss matrices to horizon adjusted loss for risk position 1, 2. Sum them up, and we will have the liquidity horizon adjusted loss for the portfolio.
\\Step 7:\\
Obtain FRTB expected shortfall attributed to each RF by calculating quadratic mean of expected shortfall of horizon adjusted loss of 5 LH.

\subsection*{Step 8-11}
Step 8:\\
we defined the FRTB ES capital charge for RFi as $IMCC(X(i)) = \frac {ES^{R,S}(X(i))} {ES^{R,C}(X (i))} ES^{F,C}(X(i))$, with $1 \leq i \leq 5$. In this case we assume that $\frac{ES^{R,S}(X(i))} {ES^{R,C}(X(i))} = 2$ for all $1 \leq i \leq 5$, and also we use the $ES(X(i))$ in task 7 as $ES^{F,C}(X(i))$.

\noindent Step 9:\\
we constructed the unconstrained portfolio with each liquidity horizon for risk position 1 and risk position 2 respectively by 
$$
X_n(6,j)=\sum^5_{i=1}X_n(i,j),\quad 1 \leq j \leq 5
$$
\noindent for n equals to 1 and 2.\\ 
\noindent Then, we defined the unconstrained portfolio for the portfolio by summing the unconstrained portfolio for risk position 1 and risk position 2 for each liquidity horizon. 
$$
X(6,j) = X_1(6,j) + X_2(6,j)
$$
\noindent And The FRTB ES capital charge for X(6) is calculated by the same way in task 8 by:
$$
IMCC(X(i)) = \frac{ES^{R,S}(X(i))} {ES^{R,C}(X (i))} ES^{F,C}(X(i))
$$
\noindent for i =6 \\
\noindent Step 10:
\noindent For the portfolio loss X, we defined the aggregate capital charge for modellable risk factors by:
$$
IMCC(X) = 0.5 IMCC(x(6)) + 0.5 \sum^5_{i=1} IMCC(X(i))
$$
\subsection*{Step 12-13}
Step 12:\\
To execute the algorithm of Euler allocation of loss for risk position 1, 2, we first find the index where loss is greater than VaR by boolean index. Then match this boolean index to loss for risk position 1, 2 and take the mean.
\\Step 13:\\
Apply the Bayes formula to calculate Euler allocation of FRTB ES.
Use the sum back methodology and get exactly the expected shortfall attributed to each RF, which verifies our calculation.

\subsection*{Step 14-15}
For step 14, we calculated the $ES^{F,C}(X_n(6,j)|X(6))$ using the same method as step 13.
\\For step 15, with the assumption in item 8 that $\frac {ES^{R,S}(X(i))} {ES^{R,C}(X(i))} = 2$, we have $IMCC(X_n(i,j)|X(i))=ES^{F,C}(X_n(i,j)|X(i))$.
\subsection*{Step 16-17}
Set two loops i from 1 to 6, k from 1 to 5. Calculated the sum of k's $IMCC(\tilde X(i,k)|X_n(i,j))$ (16) to get the Euler allocation of IMCC. Since in python, we let k start from 0, so we have 1/(5-(j+1)+1) and j in range (1,k+1). 
%-----------------------------------------------
%-----------------------------------------------
%-----------------------------------------------
%-----------------------------------------------
%-----------------------------------------------
%-----------------------------------------------
%-----------------------------------------------
\section*{\underline{Result}}

\subsection*{Step 4-5}
For step 4, we found that the $VaR_{0.99}$ for the 10 days loss of the portfolio is around: 1.958 and the $ES_{0.975}$ for the 10 days loss of the portfolio is also around: 1.968.
\\For step 5, the Euler allocation of $VaR_{0.99}$ for the risk position 1 is around: 1.102, and the Euler allocation of $VaR_{0.99}$ for the risk position 2 is around: 0.855. The Euler allocation of $ES_{0.975}$ for the risk position 1 is around: 1.141, and the Euler allocation of $ES_{0.975}$ for the risk position 2 is around: 0.827.
\subsection*{Step 6-7}
The result is:\\
0.95221332, 0.95054998, 0.95205721, 0.95254039, 0.95144739 with respect to 5 RF.

\subsection*{Step 8-11}
For task 8, we found that the FRTB ES capital charge for each factor i in RF are:

\noindent for i = 1, IMCC(X(1))= 1.9044266316512475\\
for i = 2, IMCC(X(2))  = 1.9010999601957776\\
for i = 3, IMCC(X(3))  = 1.9041144158790906\\
for i = 4, IMCC(X(4))  = 1.9050807752747572\\
for i = 5, IMCC(X(5))  = 1.9028947797673028\\

\noindent For task 9, we constructed the unconstrained portfolio for risk position 1, risk position 2, and the portfolio. Then, we found the FRTB ES capital charge for unconstrained portfolio for the portfolio($IMCC(X(6))$) is:
7.042635255965385
With the expected shortfall for each factor in liquidity horizons:
[1.96811812, 1.60070078, 1.74400856, 1.22164945, 1.19580629],

\noindent For task 10, For the portfolio loss X, its aggregate capital charge for modellable risk factors is
8.280125909366781.


\subsection*{Step 12-13}
Step 12:\\
The Euler allocation of loss for risk position 1:\\
\hrule \vspace{5pt}
\noindent 0.2395761  0.19575744 0.21324066 0.15107842 0.15275208
0.23897585 0.19498911 0.21316079 0.1520811\\  0.15323127
0.23983391 0.19533927 0.2135622  0.15187831 0.15322169
0.23901384 0.19376588\\ 0.21300172 0.15104745 0.15318629
0.23904431 0.19485251 0.21386203 0.15120541 0.1528519 \\ 
\hrule \vspace{5pt}
\noindent The Euler allocation of loss for risk position 2:\\
\hrule \vspace{5pt}
\noindent 0.24858784 0.2157392  0.25642818 0.20256674 0.23892676
0.24828428 0.21490461 0.25562263 0.20137282\\ 0.23844682
0.24794592 0.21578367 0.25661238 0.20202529 0.23810822
0.24925457 0.21722997\\ 0.25726635 0.20277242 0.23880643
0.24874853 0.21607067 0.25557722 0.20258935 0.23816956 \\
\hrule \vspace{5pt}
\vspace{15pt}
\noindent Step 13:\\
The Euler allocation of FRTB ES for risk position 1:
\begin{table}[H]
\centering
  \label{tab:my-tab1}
  \resizebox{.85\columnwidth}{!}{%
  \begin{tabular}{llllll}
  \hline
  \diagbox{i}{j}     & 1          & 2          & 3         & 4          & 5          \\ \hline
1&0.12282236 &0.08459659 &0.10517925 &0.05610977 &0.06283269 \\
2&0.1225018  &0.08408319 &0.10512528 &0.0565504  &0.06313997\\
3&0.12287793 &0.08435304 &0.10546856 &0.05645732 &0.06298003\\
4&0.12251826 &0.08360531 &0.10515932 &0.05610671 &0.06304014\\
5&0.12255516 &0.08415587 &0.10551905 &0.05622592 &0.06281875\\
 \end{tabular}%
  }
  \end{table}


\noindent The Euler allocation of FRTB ES for risk position 2:
\begin{table}[H]
\centering
  \label{tab:my-tab2}
  \resizebox{.85\columnwidth}{!}{%
  \begin{tabular}{llllll}
  \hline
  \diagbox{i}{j}     & 1          & 2          & 3         & 4          & 5          \\ \hline
1&0.12744237& 0.0932317 & 0.12648114& 0.07523228& 0.09827958\\
2&0.12727341& 0.09267116& 0.12606634& 0.07487922& 0.09825361\\
3&0.12703409& 0.09318152& 0.12672907& 0.07509833& 0.09787168\\
4&0.12776764& 0.0937295 & 0.12701285& 0.07532   & 0.09827505\\
5&0.12753039& 0.09331989& 0.12610123& 0.0753331 & 0.09788242\\
  \end{tabular}%
  }
  \end{table}

\subsection*{Step 14-15}
For step 14:\\ $ES^{F,C}(X_1(6,j)|X(6))$ is:
\begin{table}[H]
  \label{tab:my-table}
  \resizebox{\columnwidth}{!}{%
  \begin{tabular}{llllll}
  \hline
  j                         & 1          & 2          & 3         & 4          & 5          \\ \hline
  $ES^{F,C}(X_1(6,j)|X(6))$ &0.63757743& 0.41627193& 0.48337935& 0.226784  & 0.19917252 \\ \hline
  \end{tabular}%
  }
  \end{table}
\noindent$ES^{F,C}(X_2(6,j)|X(6))$ is:
\begin{table}[H]
  \label{tab:my-table2}
  \resizebox{\columnwidth}{!}{%
  \begin{tabular}{llllll}
  \hline
  j                         & 1          & 2          & 3         & 4          & 5          \\ \hline
  $ES^{F,C}(X_2(6,j)|X(6))$ &0.46242761& 0.31136153& 0.38037363& 0.19703998& 0.20690968 \\ \hline
  \end{tabular}%
  }
  \end{table}
\noindent For step 15:\\
\noindent $IMCC(X_1(i,j)|X(i))$ is:
\begin{table}[H]
\centering
  \label{tab:my-table3}
  \resizebox{.85\columnwidth}{!}{%
  \begin{tabular}{llllll}
  \hline
  \diagbox{i}{j}     & 1          & 2          & 3         & 4          & 5          \\ \hline
  1&0.12282236 &0.08459659 &0.10517925 &0.05610977 &0.06283269  \\
  2&0.1225018  &0.08408319 &0.10512528 &0.0565504  &0.06313997  \\
  3&0.12287793 &0.08435304 &0.10546856 &0.05645732 &0.06298003  \\
  4&0.12251826 &0.08360531 &0.10515932 &0.05610671 &0.06304014  \\
  5&0.12255516 &0.08415587 &0.10551905 &0.05622592 &0.06281875  \\
  6&0.63757743 &0.41627193 &0.48337935 &0.226784   &0.19917252  \\
  \end{tabular}%
  }
  \end{table}

\noindent $IMCC(X_2(i,j)|X(i))$ is:
\begin{table}[H]
\centering
  \label{tab:my-table4}
  \resizebox{.85\columnwidth}{!}{%
  \begin{tabular}{llllll}
  \hline
  \diagbox{i}{j}     & 1          & 2          & 3         & 4          & 5          \\ \hline
1&0.12744237 &0.0932317  &0.12648114 &0.07523228 &0.09827958\\
2&0.12727341 &0.09267116 &0.12606634 &0.07487922 &0.09825361\\
3&0.12703409 &0.09318152 &0.12672907 &0.07509833 &0.09787168\\
4&0.12776764 &0.0937295  &0.12701285 &0.07532    &0.09827505\\
5&0.12753039 &0.09331989 &0.12610123 &0.0753331  &0.09788242\\
6&0.46242761 &0.31136153 &0.38037363 &0.19703998 &0.20690968\\
  \end{tabular}%
  }
  \end{table}
\subsection*{Step 16-17}
\noindent $IMCC(\tilde{X_1}(i,k)|X(i))$ is:
\begin{table}[H]
\centering
  \label{tab:my-table5}
  \resizebox{.85\columnwidth}{!}{%
  \begin{tabular}{llllll}
  \hline
  \diagbox{i}{j}     & 1          & 2          & 3         & 4          & 5          \\ \hline
1&0.02456447 &0.04571362 &0.08077337 &0.10882825 &0.17166094\\
2&0.02450036 &0.04552116 &0.08056292 &0.10883812 &0.17197809\\
3&0.02457559 &0.04566385 &0.08082004 &0.1090487  &0.17202873\\
4&0.02450365 &0.04540498 &0.08045809 &0.10851144 &0.17155158\\
5&0.02451103 &0.04555    &0.08072301 &0.10883597 &0.17165472\\
6&0.12751549 &0.23158347 &0.39270992 &0.50610192 &0.70527444\\
  \end{tabular}%
  }
  \end{table}
  
\newpage
\noindent $IMCC(\tilde{X_2}(i,k)|X(i))$ is:
\begin{table}[H]
\centering
  \label{tab:my-table6}
  \resizebox{.85\columnwidth}{!}{%
  \begin{tabular}{llllll}
  \hline
  \diagbox{i}{j}     & 1          & 2          & 3         & 4          & 5          \\ \hline
1&0.02548847 &0.0487964  &0.09095678 &0.12857292 &0.2268525 \\
2&0.02545468 &0.04862247 &0.09064458 &0.12808419 &0.2263378 \\
3&0.02540682 &0.0487022  &0.09094522 &0.12849439 &0.22636607\\
4&0.02555353 &0.0489859  &0.09132352 &0.12898352 &0.22725857\\
5&0.02550608 &0.04883605 &0.09086979 &0.12853634 &0.22641876\\
6&0.09248552 &0.1703259  &0.29711711 &0.3956371  &0.60254678\\
  \end{tabular}%
  }
  \end{table}
\end{document}
